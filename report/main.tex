\documentclass[a4paper]{article}

% Global layout
\usepackage{fancyhdr, graphicx, hyperref, indentfirst, lastpage, setspace}
\usepackage[margin=40mm]{geometry}

% Encoding
\usepackage[utf8]{vntex, inputenc} % vntex first to avoid Vietnamese auto-captions
\usepackage{amsmath, amssymb, gensymb}

% Better table
\usepackage{array, booktabs, multicol, multirow, siunitx, tabularx}

% Code space
\usepackage[dvipsnames]{xcolor}
\usepackage{tikz}
\usepackage[framemethod=tikz]{mdframed}
\usepackage{minted, verbatim} % needs --shell-escape flag and Pygments

% Graphics
\usepackage{caption, float}

% Page setup
% \allowdisplaybreaks{} % to have page breaks inside align* environment
\hypersetup{urlcolor=blue,linkcolor=black,citecolor=red,colorlinks=true}
\usemintedstyle{emacs}
\numberwithin{equation}{section}
% \renewcommand{\arraystretch}{1.2} % space between table rows

% Global style setup
\makeatletter % change font size for not having underfull hbox
\renewcommand\Huge{\@setfontsize\Huge{22pt}{18}}
\makeatother

\AtBeginDocument{\renewcommand*\contentsname{Contents}}
\AtBeginDocument{\renewcommand*\refname{References}}
\setlength{\headheight}{40pt}
\pagestyle{fancy}
\fancyhead{} % clear all header fields
\fancyhead[L]{
  \begin{tabular}{rl}
    \begin{picture}(25,15)(0,0)
    \put(0,-8){\includegraphics[width=8mm, height=8mm]{hcmut.png}}
    \end{picture}
    \begin{tabular}{l}
      \textbf{\bf \ttfamily University of Technology, Ho Chi Minh City}\\
      \textbf{\bf \ttfamily Faculty of Computer Science and Engineering}
    \end{tabular}
  \end{tabular}
}
\fancyhead[R]{
	\begin{tabular}{l}
		\tiny \bf \\
		\tiny \bf
	\end{tabular}  }
\fancyfoot{} % clear all footer fields
\fancyfoot[L]{\scriptsize \ttfamily Report for Advanced Programming --- Academic year 2020--2021}
\fancyfoot[R]{\scriptsize \ttfamily Page {\thepage}/\pageref{LastPage}}
\renewcommand{\headrulewidth}{0.3pt}
\renewcommand{\footrulewidth}{0.3pt}

% \everymath{\color{blue}}

\newcommand*\mean[1]{\bar{#1}}
\newenvironment{code}[1]
{\VerbatimEnvironment%
  \begin{mdframed}[leftline=false,rightline=false,backgroundcolor=magenta!10,nobreak=false]%
    \begin{minted}[linenos=true,breaklines,breaksymbolleft=,obeytabs=true,tabsize=2]{#1}%
}
{
    \end{minted}%
  \end{mdframed}%
}
\newenvironment{console}
{\VerbatimEnvironment%
  \begin{mdframed}[leftline=false,rightline=false,backgroundcolor=teal!10,nobreak=false]%
    \begin{minted}[linenos=true,breaklines,breaksymbolleft=,obeytabs=true,tabsize=2]{text}%
}
{
    \end{minted}%
  \end{mdframed}%
}

\begin{document}

\begin{titlepage}
  \begin{center}
    VIETNAM NATIONAL UNIVERSITY, HO CHI MINH CITY \\
    UNIVERSITY OF TECHNOLOGY \\
    FACULTY OF COMPUTER SCIENCE AND ENGINEERING
  \end{center}

  \vspace{1cm}

  \begin{figure}[H]
    \centering
    \includegraphics[width=0.5\textwidth]{hcmut.png}
  \end{figure}

  \vspace{1cm}

  \begin{center}
    \begin{tabular}{c}
      \textbf{\Large Advanced Programming (CO2039)}             \\
      {}                                                        \\
      \midrule                                                  \\
      \textbf{\Large Report (Semester 202, Duration: 01 weeks)} \\
      {}                                                        \\
      \textbf{\Huge OOP vs FP}                                  \\
      {}                                                        \\
      \bottomrule
    \end{tabular}
  \end{center}

  \vspace{3cm}

  \begin{table}[h]
    \begin{tabular}{rl}
      \hspace{1cm} Advisor:      & Mr.\ Lê Lam Sơn \\
                                 &                 \\
      \hspace{1cm} Student Name: & Nguyễn Hoàng    \\
      \hspace{1cm} Student ID\@: & 1952255         \\
    \end{tabular}
  \end{table}

  \begin{center}
    {\footnotesize HO CHI MINH CITY, AUGUST 2021}
  \end{center}
\end{titlepage}


% \thispagestyle{empty}

% \newpage
% \tableofcontents
% \newpage

\section{OOP and FP in baking a pizza}
OOP makes code understandable by encapsulating moving parts.
FP makes code understandable by minimizing moving parts.

What? Alright that sounds a bit rough, let's rephrase this a bit.
OOP aims to model the world in self-contained entities, and affects change by modifying the state of itself or other entities.
FP on the other hand aims to not modify the original data, but rather creates new data given some existing data.

To demonstrate this, we will try to make a pizza.
With OOP, a big box or object with all the materials to create a pizza is available, and the helper methods will slowly transform them into a complete pizza.
FP will take a different approach, as materials are given to each stage/step/activity in order to be used in the next activity until the final product is achieved.

We will try to describe this pizza making progress programmatically using C++ and Haskell.

Let's start with a complete C++ program
\begin{mdframed}[leftline=false,rightline=false,backgroundcolor=magenta!10,nobreak=false]%
  \inputminted[linenos=true,breaklines,breaksymbolleft=,obeytabs=true,tabsize=2]{C}{../cpp/main.cpp}
\end{mdframed}

Output of this program
\begin{mdframed}[leftline=false,rightline=false,backgroundcolor=teal!10,nobreak=false]%
  \inputminted[linenos=true,breaksymbolleft=,obeytabs=true,tabsize=2]{text}{../cpp/main.txt}
\end{mdframed}

Nice! Let's do this again, but with Haskell
\begin{mdframed}[leftline=false,rightline=false,backgroundcolor=magenta!10,nobreak=false]%
  \inputminted[linenos=true,breaklines,breaksymbolleft=,obeytabs=true,tabsize=2]{Haskell}{../hs/main.hs}
\end{mdframed}

Output of this program
\begin{mdframed}[leftline=false,rightline=false,backgroundcolor=teal!10,nobreak=false]%
  \inputminted[linenos=true,breaksymbolleft=,obeytabs=true,tabsize=2]{text}{../hs/main.txt}
\end{mdframed}

\newpage
\section{Comparing the pizza-making methods}
With the pizza making out of the way, there are definitely some things noticeable between the two approaches.
Obviously, the procedure of the process does not change, but the way the materials or the pizza, otherwise known as the data, are handled and processed is different.

\begin{center}
  \begin{tabularx}{\textwidth}{l*{2}{X}}
    \toprule
         & OOP                                                                                                                                 & FP                                                                         \\
    \cmidrule(lr){2-3}
    Pros & OOP objects are self contained, meaning that an object has certain characteristics (attributes) and things that it can do (methods) & FP doesn't care about the data, but rather what it can do with the data    \\
         & Any changes that are applied is reflected on the object itself                                                                      & Data in FP is not intended to change, if it does, new data is just created \\
         &                                                                                                                                     &                                                                            \\
    \midrule
    Cons &                                                                                                                                     &                                                                            \\
         &                                                                                                                                     &                                                                            \\
         &                                                                                                                                     &                                                                            \\
    \bottomrule
  \end{tabularx}
\end{center}

\newpage
\section*{Easter egg}

\end{document}
